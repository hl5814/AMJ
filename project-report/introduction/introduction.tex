\chapter{Introduction}
JavaScript is highly dynamic, expressive language that supports real time evaluation and dynamic code generation at runtime. Today, JavaScript are not only been used in the development of large-scale web site such as Facebook but also used in different browser extensions. However the expressiveness and dynamic feature of JavaScript is often misused by attackers. The most widely used technique is the use of \textbf{eval} and \textbf{document.write} to convert runtime strings into invocable code. \\ \\
In recently years, the number of JavaScript based malware attacks have increased. By exploiting numerous vulnerabilities in various web applications, attackers can launch a wide range of at­tacks such as cross-site scripting(XSS)\cite{XSS}, cross-site request forgery(CSRF)\cite{CSRF}, drive-by downloads \cite{Drive-by Downloads}, etc. And most Internet users rely on the anti-virus software. However,  most existing anti-virus software use static signature to prevent malicious JavaScript being executed. But attacker usually applies different JavaScript obfuscation techniques in order to hide the malicious content to evade detection. \\ \\
Apart from the language features, \textit{exploit kits}\cite{EK} allow attackers to create new exploits by using multiple existing exploits. Depending on the change, this can be easily accomplished within minutes. However, this also means there exists certain patterns behind the exploits. 
\section{Objectives}
In order to provide a reliable detection for malicious JavaScript, this project is to build a malicious JavaScript code Analyser that can not only classify a piece of JavaScript code into an known malicious class but also analysis the report malicious JavaScript patterns used. \\ \\
Malware cycle is an asymmetry loop between attackers and defenders. Once attackers develop the new malicious codes, they can always test it with the existing anti-virus engines until it can evade the detection. This project should be able to provide information about the relations between different patterns. In order to let the defenders to have a better understanding on how the patterns evolve with time. 
\section{Challenges}
JavaScript is embedded in web pages, except the script blocks, JavaScript codes can also exists in event handlers (e.g. button onclick event). Some malicious content may only be generated based on those events which means the malicious content can be hidden on the static perspective. So correctly detect a piece of JavaScript is malicious or not is the biggest challenge.
\section{Contributions}
...